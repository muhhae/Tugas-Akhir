\begin{center}
  \large\textbf{ABSTRACT}
\end{center}

\addcontentsline{toc}{chapter}{ABSTRACT}

\vspace{2ex}

\begingroup
% Menghilangkan padding
\setlength{\tabcolsep}{0pt}

\noindent
\begin{tabularx}{\textwidth}{l >{\centering}m{3em} X}
  \emph{Name}     & : & \name{}         \\

  \emph{Title}    & : & \engtatitle{}   \\

  \emph{Advisors} & : & 1. \advisor{}   \\
                  &   & 2. \coadvisor{} \\
\end{tabularx}
\endgroup

% Ubah paragraf berikut dengan abstrak dari tugas akhir dalam Bahasa Inggris
This research focuses on the development and analysis of a machine
learning-based Clock cache eviction algorithm. The study aims to
improve the performance of the Clock algorithm by utilizing a
supervised learning model in the decision-making process for cache eviction.
Simulations were conducted using custom software based on LibCacheSim
with a trace replay data access scenario. The performance of the
developed algorithm was analyzed using metrics such as hit ratio,
miss ratio, and the number of promotions.
The results indicate that integrating machine learning into the Clock
algorithm can enhance performance by reducing the number of cache
promotions, making it more optimal than the conventional Clock algorithm.

% Ubah kata-kata berikut dengan kata kunci dari tugas akhir dalam Bahasa Inggris
\emph{Keywords}: \emph{Cache, Clock, LRU, FIFO, System, Machine Learning, Supervised Learning}
