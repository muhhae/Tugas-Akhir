\begin{center}
  \large\textbf{ABSTRAK}
\end{center}

\addcontentsline{toc}{chapter}{ABSTRAK}

\vspace{2ex}

\begingroup
% Menghilangkan padding
\setlength{\tabcolsep}{0pt}

\noindent
\begin{tabularx}{\textwidth}{l >{\centering}m{2em} X}
  Nama Mahasiswa    & : & \name{}         \\

  Judul Tugas Akhir & : & \tatitle{}      \\

  Pembimbing        & : & 1. \advisor{}   \\
                    &   & 2. \coadvisor{} \\
\end{tabularx}
\endgroup

% Ubah paragraf berikut dengan abstrak dari tugas akhir
Penelitian ini berfokus pada pengembangan dan analisis algoritma \textit{cache eviction} Clock berbasis \textit{machine learning}. Studi ini bertujuan untuk meningkatkan kinerja algoritma Clock dengan memanfaatkan model \textit{supervised learning} dalam proses pengambilan keputusan untuk \textit{cache eviction}. Simulasi dilakukan menggunakan perangkat lunak kustom berbasis LibCacheSim dengan skenario akses data \textit{trace replay}. Kinerja algoritma yang dikembangkan dianalisis menggunakan metrik seperti \textit{hit ratio}, \textit{miss ratio}, dan jumlah promosi. Hasil penelitian menunjukkan bahwa mengintegrasikan \textit{machine learning} ke dalam algoritma Clock dapat meningkatkan kinerja dengan mengurangi jumlah promosi \textit{cache}, menjadikannya lebih optimal dibandingkan algoritma Clock konvensional.

% Ubah kata-kata berikut dengan kata kunci dari tugas akhir
Kata Kunci: \textit{Cache, Clock, LRU, FIFO, Sistem, Machine Learning, Supervised Learning}
