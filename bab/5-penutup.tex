\chapter{PENUTUP}
\label{chap:penutup}

Bab ini merangkum keseluruhan hasil dari penelitian yang telah
dilakukan. Bagian pertama, Kesimpulan, menyajikan ringkasan temuan
utama, sementara bagian kedua, Saran, memberikan rekomendasi untuk
pengembangan penelitian ini di masa mendatang.

\section{Kesimpulan}
\label{sec:kesimpulan}

Berdasarkan perancangan, implementasi, dan pengujian yang telah
dilakukan dalam penelitian ini, dapat ditarik beberapa kesimpulan
sebagai berikut:

\begin{enumerate}[nolistsep]
  \item Integrasi model \textit{machine learning} ke dalam algoritma
    \textit{cache eviction} CLOCK terbukti dapat meningkatkan
    efisiensi manajemen \textit{cache}. Model yang dikembangkan mampu
    mengidentifikasi dan mencegah promosi objek yang tidak perlu atau sia-sia.
  \item Penerapan model regresi logistik yang telah disesuaikan
    berhasil mengurangi jumlah total promosi objek sekitar
    \textbf{10\%} pada data pengujian. Pengurangan ini dicapai tanpa
    memberikan dampak negatif yang signifikan terhadap \textit{miss
    ratio} sistem, yang berarti kinerja utama \textit{cache} dalam
    melayani permintaan data tetap terjaga.
  \item Penelitian ini menunjukkan bahwa bahkan model \textit{machine
    learning} yang relatif sederhana dan ringan dapat memberikan
    peningkatan kinerja yang berarti pada algoritma \textit{cache}
    klasik. Hal ini membuka peluang untuk implementasi serupa pada
    sistem nyata di mana \textit{overhead} komputasi menjadi
    pertimbangan penting.
\end{enumerate}

Secara keseluruhan, penelitian ini berhasil menunjukkan bahwa
pendekatan berbasis data menggunakan \textit{machine learning} adalah
arah yang menjanjikan untuk mengoptimalkan algoritma CLOCK,
menjadikannya lebih cerdas dan lebih efisien dalam mengelola sumber
daya \textit{cache}.

\section{Saran}
\label{sec:saran}

Meskipun penelitian ini telah mencapai tujuannya, terdapat beberapa
area yang dapat dieksplorasi lebih lanjut untuk pengembangan di masa
depan. Beberapa saran untuk penelitian selanjutnya antara lain:

\begin{enumerate}[nolistsep]
  \item \textbf{Eksplorasi Model Machine Learning yang Lebih
    Kompleks:} Penelitian ini menggunakan model regresi logistik.
    Studi di masa depan dapat mengeksplorasi penggunaan arsitektur
    model yang lebih canggih, seperti \textit{neural network} yang
    lebih dalam atau model berbasis \textit{ensemble} (misalnya,
    Gradient Boosting atau Random Forest) untuk melihat apakah
    peningkatan kompleksitas model dapat memberikan akurasi dan
    efisiensi yang lebih tinggi.
  \item \textbf{Penerapan Reinforcement Learning:} Mengingat sifat
    dinamis dari manajemen \textit{cache}, pendekatan menggunakan
    \textit{reinforcement learning} (RL) dapat menjadi alternatif
    yang sangat menarik. Model RL dapat dilatih untuk membuat
    kebijakan \textit{eviction} yang adaptif secara
    \textit{real-time} terhadap perubahan pola beban kerja.
  \item \textbf{Pengujian pada Lingkungan Produksi Nyata:} Penelitian
    ini terbatas pada lingkungan simulasi. Langkah selanjutnya yang
    ideal adalah mengimplementasikan dan menguji algoritma yang
    diusulkan pada sistem produksi nyata untuk memvalidasi kinerjanya
    di bawah beban kerja yang sesungguhnya dan mengukur
    \textit{overhead} inferensi model secara akurat.
  \item \textbf{Analisis Fitur yang Lebih Mendalam:} Melakukan
    analisis yang lebih mendalam terhadap fitur-fitur (metadata
    objek) yang paling berpengaruh terhadap keputusan promosi dapat
    membantu dalam merancang model yang lebih efisien dan ringan di masa depan.
\end{enumerate}

