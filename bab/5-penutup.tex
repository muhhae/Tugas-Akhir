\chapter{PENUTUP}

Bab ini merangkum temuan-temuan kunci yang diperoleh dari eksperimen integrasi
\textit{machine learning} ke dalam algoritma manajemen \textit{cache} CLOCK,
serta memberikan rekomendasi strategis untuk pengembangan penelitian
selanjutnya.

\section{Kesimpulan}

Berdasarkan analisis data pengujian dan evaluasi kinerja yang telah
dipaparkan pada bab sebelumnya,
penulis menarik beberapa kesimpulan kritis sebagai berikut:

\begin{enumerate}
  \item \textbf{Efisiensi Penulisan Terbukti Namun Terbatas.}
    Implementasi ML-CLOCK berhasil membuktikan hipotesis awal bahwa
    model prediktif
    dapat mengurangi operasi penulisan (\textit{promotions}) ke memori utama.
    Pengurangan konsisten sebesar 8--10\% tercapai di seluruh variasi
    ukuran \textit{cache}.
    Meskipun angka ini signifikan untuk memperpanjang umur perangkat
    penyimpanan (\textit{endurance}),
    capaian ini masih jauh di bawah batas optimal teoretis
    \textit{Offline CLOCK} yang mencapai 43\%.
    Hal ini menunjukkan bahwa model masih bersikap konservatif dalam
    pengambilan keputusan.

  \item \textbf{Trade-off Kinerja pada Skala Besar.}
    Terdapat pola kinerja non-linear yang jelas. Pada lingkungan
    dengan tekanan memori tinggi
    (ukuran \textit{cache} 1\% -- 20\%), ML-CLOCK memberikan keuntungan ganda:
    pengurangan \textit{miss ratio} dan penghematan promosi.
    Namun, terjadi regresi kinerja pada ukuran \textit{cache} 40\%,
    di mana model menjadi terlalu selektif (\textit{over-selective})
    sehingga membuang objek
    yang seharusnya disimpan saat ruang tersedia.
    Ini menegaskan bahwa pendekatan ML statis tanpa adaptasi dinamis terhadap
    ketersediaan ruang memiliki risiko pada kapasitas besar.

  \item \textbf{Superioritas Oracle Mengungkap Batas Prediksi Lokal.}
    Dominasi strategi \textit{Optimal Zipf Promotion} terhadap
    ML-CLOCK di seluruh metrik
    mengkonfirmasi bahwa dalam distribusi Zipfian, pengetahuan global
    mengenai popularitas objek
    jauh lebih berharga daripada pola akses lokal (\textit{recency}
    dan frekuensi jangka pendek).
    ML-CLOCK yang hanya mengandalkan metadata lokal kesulitan untuk
    mereplikasi akurasi
    yang dimiliki oleh algoritma berbasis pengetahuan global tersebut.

  \item \textbf{Kontribusi Arsitektur Integrasi ONNX.}
    Terlepas dari kinerja model prediksi spesifik yang digunakan,
    kontribusi teknis utama dari penelitian ini adalah keberhasilan
    pembangunan kerangka kerja
    (\textit{framework}) simulasi hibrida.
    Mekanisme integrasi model Python ke dalam lingkungan C++ berkinerja tinggi
    menggunakan format \textbf{ONNX (Open Neural Network Exchange)}
    terbukti andal,
    stabil, dan minim latensi.
    Infrastruktur ini memecahkan hambatan bahasa pemrograman yang
    selama ini menyulitkan
    peneliti sistem operasi untuk mengadopsi model \textit{deep
    learning} modern.
\end{enumerate}

\section{Saran}

Penelitian ini membuka jalan bagi eksplorasi lebih lanjut dalam
domain \textit{AI-driven cache replacement}.
Untuk mengatasi keterbatasan yang ditemukan, penulis mengajukan
saran-saran berikut:

\begin{enumerate}
  \item \textbf{Pemanfaatan Kerangka Kerja ONNX untuk Riset Lanjutan.}
    Arsitektur simulasi yang telah dibangun dengan integrasi
    \textit{ONNX Runtime}
    dirancang agar bersifat modular dan agnostik terhadap jenis model.
    Peneliti lain sangat disarankan untuk menggunakan kembali
    (\textit{reuse}) infrastruktur ini
    untuk menguji arsitektur model yang berbeda (seperti LSTM,
    Transformer, atau Reinforcement Learning)
    tanpa perlu membangun ulang simulator dari nol.
    Kode sumber simulator dapat dijadikan basis standar (\textit{testbed}) untuk
    komunitas riset manajemen memori.

  \item \textbf{Penggunaan Trace Dunia Nyata (Real-world Workloads).}
    Evaluasi pada dataset sintetis Zipfian memiliki keterbatasan dalam menangkap
    dinamika temporal yang kompleks, seperti \textit{burstiness} atau
    perubahan fase popularitas.
    Penelitian selanjutnya disarankan menggunakan \textit{trace}
    produksi dari CDN
    (Content Delivery Network) atau basis data skala besar.
    Data riil akan memberikan tantangan yang lebih adil bagi model ML
    dibandingkan distribusi statis Zipfian yang cenderung
    menguntungkan metode heuristik sederhana.

  \item \textbf{Eksplorasi Fitur Non-Linear.}
    Fitur \texttt{vtime\_since\_access} terbukti kurang efektif dan
    berpotensi menjadi \textit{noise}
    pada distribusi Zipfian. Disarankan untuk mengeksplorasi fitur
    yang lebih kompleks,
    seperti distribusi waktu antar-kedatangan (\textit{inter-arrival
    time distribution})
    atau fitur berbasis grafik relasi antar-objek, untuk menangkap pola akses
    yang tidak terdeteksi oleh statistik frekuensi sederhana.

  \item \textbf{Mekanisme Threshold Adaptif.}
    Untuk mengatasi masalah regresi pada \textit{cache} ukuran besar,
    model statis dengan ambang batas tetap (0.6 -- 0.8) harus ditinggalkan.
    Disarankan pengembangan mekanisme \textit{adaptive thresholding}
    di mana ambang batas penerimaan objek diturunkan secara otomatis
    ketika tekanan pada memori rendah (banyak ruang kosong) dan dinaikkan
    ketika memori penuh, menyerupai perilaku kontrol umpan balik
    (\textit{feedback loop}).
\end{enumerate}
