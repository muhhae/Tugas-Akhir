\chapter{PERANCANGAN DAN IMPLEMENTASI SISTEM}
\label{chap:perancangan_implementasi}

Bab ini merinci metodologi yang digunakan dalam penelitian ini, mulai
dari perancangan dan pengembangan perangkat lunak simulasi, proses
pengumpulan dan pengolahan data, hingga implementasi dan evaluasi
model \textit{machine learning}.

\section{Metodologi Penelitian}
\label{sec:metodologi_penelitian}

\subsection{Pengembangan Perangkat Lunak Simulasi}
Langkah pertama dalam penelitian ini adalah pengembangan perangkat
lunak simulasi kustom yang mampu memenuhi kebutuhan analisis.
Perangkat lunak ini dirancang dengan beberapa persyaratan utama:
\begin{enumerate}
  \item \textbf{Integrasi Model Machine Learning:} Perangkat lunak
    harus mampu memanggil dan menjalankan model \textit{machine
    learning} secara efisien di dalam fungsi \textit{eviction} dari
    algoritma \textit{cache}.
  \item \textbf{Dukungan Algoritma Pembanding:} Selain algoritma yang
    diusulkan, perangkat lunak harus dapat menjalankan algoritma
    CLOCK standar dan Offline CLOCK sebagai tolok ukur perbandingan kinerja.
  \item \textbf{Pengumpulan Data dari Offline CLOCK:} Perangkat lunak
    harus mampu menjalankan mode Offline CLOCK untuk mengumpulkan
    data-data krusial, antara lain:
    \begin{enumerate}
      \item Nilai \textit{miss ratio} dan jumlah promosi untuk setiap
        iterasi simulasi.
      \item Metadata dari setiap objek pada saat keputusan promosi
        dibuat, beserta label apakah promosi tersebut pada akhirnya
        sia-sia (\textit{wasted}) atau tidak.
      \item Kemampuan untuk mengekspor semua data yang terkumpul ke
        dalam format CSV (\textit{Comma-Separated Values}) untuk
        memfasilitasi proses pengolahan dan analisis data lebih lanjut.
    \end{enumerate}
\end{enumerate}
Untuk mempercepat proses pengembangan, penulis memanfaatkan
\textbf{LibCacheSim}, sebuah pustaka sumber terbuka yang dirancang
khusus untuk pengembangan simulator \textit{cache}. Pustaka ini
ditulis dalam bahasa C/C++, yang menjamin kinerja komputasi yang
sangat efisien, terutama untuk tugas-tugas berat seperti pengumpulan
data dari \textit{trace} berukuran besar.

\begin{figure}[H]
  \centering
  \includegraphics[width=0.6\textwidth]{gambar/alur-program.png}
  \caption{Diagram alur dari program simulasi yang dikembangkan.}
  \label{fig:alur_program}
\end{figure}

\subsection{Pengumpulan Data}
Pengumpulan data awal dilakukan sepenuhnya menggunakan perangkat
lunak simulasi yang telah dikembangkan. Tujuannya adalah untuk
mendapatkan metrik kinerja dasar seperti jumlah promosi dan
\textit{miss ratio} dari algoritma CLOCK standar. Selain itu, pada
fase ini juga dilakukan pembuatan dataset yang akan digunakan untuk
melatih model \textit{machine learning}. Dataset ini dibuat dengan
cara mencatat (me-\textit{logging}) metadata dari setiap objek pada
saat keputusan promosi akan dibuat, lalu memberinya label `wasted`
atau `not wasted` berdasarkan informasi dari simulasi Offline CLOCK.

\textit{Trace} yang digunakan dalam penelitian ini adalah
\textit{trace} sumber terbuka yang juga digunakan oleh
penelitian-penelitian sebelumnya di bidang ini. \textit{Trace} ini
mencakup berbagai jenis beban kerja (\textit{workload}) yang
realistis, seperti \textit{BlockStore}, \textit{KVStorage}, CDN,
\textit{Proxy}, dan \textit{Photo Storage}, yang berasal dari
berbagai institusi teknologi terkemuka seperti Twitter, Meta,
Tencent, dan Alibaba. Selain itu, penulis juga menggunakan
\textit{trace} sintetis yang dibuat berdasarkan distribusi Zipfian
untuk mempermudah pemahaman pola data dalam lingkungan yang lebih terkontrol.

\subsection{Pengolahan dan Analisis Data}
Untuk mempermudah proses analisis, penulis mengembangkan beberapa
skrip Python. Skrip ini bertugas untuk mengolah data mentah dari
format CSV, melakukan plotting data menggunakan pustaka seperti
Plotly dan Pandas, dan menyajikannya dalam sebuah laporan HTML
interaktif. Visualisasi ini sangat membantu dalam mengidentifikasi
pola, korelasi, dan anomali dalam data sebelum tahap pengembangan model.

\begin{figure}[H]
  \centering
  \includegraphics[width=0.6\textwidth]{gambar/datasets-figures.png}
  \caption{Contoh visualisasi distribusi dari metadata \texttt{lifetime\_freq}.}
  \label{fig:distribusi_metadata}
\end{figure}

\subsection{Pengembangan Model Machine Learning}
Dengan menggunakan dataset yang telah diolah, sebuah model
\textit{machine learning} dikembangkan. Model ini dirancang untuk
menerima masukan berupa metadata objek (seperti frekuensi akses,
waktu sejak akses terakhir, ukuran objek, dll.) dan menghasilkan
keluaran berupa probabilitas (nilai antara 0 hingga 1) yang
mengindikasikan kelayakan objek tersebut untuk dipromosikan. Nilai
yang tinggi menandakan bahwa objek tersebut kemungkinan besar tidak
layak untuk dipromosikan.

\begin{figure}[H]
  \centering
  \includegraphics[width=0.4\textwidth]{gambar/ml-diagrams.png}
  \caption{Diagram proses pelatihan model \textit{machine learning}.}
  \label{fig:proses_pelatihan}
\end{figure}

\begin{figure}[H]
  \centering
  \includegraphics[width=0.8\textwidth]{gambar/model-arch.png}
  \caption{Arsitektur model \textit{machine learning} yang digunakan.}
  \label{fig:arsitektur_model}
\end{figure}

\subsection{Implementasi Model pada Perangkat Lunak Simulasi}
Model \textit{machine learning} yang telah dilatih kemudian
diimplementasikan kembali ke dalam perangkat lunak simulasi. Untuk
menjembatani antara lingkungan Python (untuk pelatihan) dan C++
(untuk simulasi), model diekspor ke dalam format \textbf{ONNX (Open
Neural Network Exchange)}. Model dalam format ONNX ini kemudian
dijalankan di dalam simulator menggunakan \textbf{ONNX Runtime},
sebuah \textit{inference engine} berkinerja tinggi. Di dalam fungsi
\textit{eviction} pada algoritma CLOCK, sebelum sebuah objek
dipromosikan, metadatanya akan diekstrak dan dijadikan masukan untuk
model. Jika keluaran dari model melebihi ambang batas
(\textit{threshold}) yang telah ditentukan, objek tersebut akan
langsung digusur alih-alih dipromosikan.

\begin{figure}[H]
  \centering
  \includegraphics[width=0.6\textwidth]{gambar/model-impl.png}
  \caption{Diagram alur implementasi model pada simulator.}
  \label{fig:implementasi_model}
\end{figure}

\subsection{Evaluasi Model}
Tahap terakhir adalah evaluasi kinerja model yang telah terintegrasi.
Kinerja algoritma CLOCK yang telah dimodifikasi dengan
\textit{machine learning} ini kemudian dibandingkan secara langsung
dengan algoritma CLOCK standar dan tolok ukur teoretis dari Offline
CLOCK. Metrik utama yang menjadi fokus perbandingan adalah jumlah
total promosi dan \textit{miss ratio}.

\section{Perangkat Keras dan Lunak}
Penelitian ini didukung oleh beberapa perangkat keras dan lunak sebagai berikut:
\subsection{Perangkat Keras}
\begin{enumerate}
  \item \textbf{Testbed Utama:} Sebuah PC server dengan prosesor
    Intel Xeon Silver 4114 (40 \textit{cores}) @ 3.000GHz dan RAM
    sebesar 192 GiB.
  \item \textbf{Laptop Pribadi:} Digunakan untuk pengembangan dan analisis awal.
\end{enumerate}
\subsection{Perangkat Lunak}
\begin{enumerate}
  \item \textbf{Python:} Bahasa pemrograman utama untuk pengolahan
    data dan pengembangan model.
  \item \textbf{Plotly \& Pandas:} Pustaka Python untuk analisis dan
    visualisasi data.
  \item \textbf{C++:} Bahasa pemrograman utama untuk perangkat lunak simulasi.
  \item \textbf{libCacheSim:} Pustaka dasar untuk simulator \textit{cache}.
  \item \textbf{ONNX Runtime:} Untuk menjalankan inferensi model di
    lingkungan C++.
  \item \textbf{SSH:} Untuk mengakses dan mengelola \textit{testbed}
    dari jarak jauh.
\end{enumerate}

