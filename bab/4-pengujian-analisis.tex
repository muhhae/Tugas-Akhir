\chapter{PENGUJIAN DAN ANALISIS}

Bagian ini menguraikan hasil evaluasi kuantitatif dari model ML-CLOCK
yang diusulkan.
Kinerja model dibandingkan dengan tiga standar:
CLOCK Standar (Baseline), Optimal Zipf Promotion (Heuristik Oracle Sederhana),
dan Offline CLOCK (Batas Atas Teoretis).
Pengujian dilakukan pada empat variasi ukuran \textit{cache}: 1\%,
10\%, 20\%, dan 40\%
dari total ukuran dataset unik.

\section{Perbandingan Kinerja Keseluruhan}

Tabel \ref{tab:comparison_all} menyajikan rangkuman lengkap kinerja
seluruh metode.
Nilai negatif pada \textit{Delta Miss Ratio} menandakan perbaikan
(penurunan miss),
sedangkan nilai positif menandakan degradasi kinerja.

\begin{table}[htbp]
  \centering
  \caption{Perbandingan Komprehensif: ML-CLOCK vs. Referensi}
  \label{tab:comparison_all}
  \resizebox{\textwidth}{!}
      & ML-CLOCK (Usulan) & -0.05\% & 10\% \\
      & Optimal Zipf & \textbf{-0.20\%} & 10\% \\
      & Offline CLOCK & -0.35\% & 40\% \\
      \midrule
      \multirow{3}{*}{10\%}
      & ML-CLOCK (Usulan) & -0.10\% & 10\% \\
      & Optimal Zipf & \textbf{-0.90\%} & 13\% \\
      & Offline CLOCK & -1.80\% & 43\% \\
      \midrule
      \multirow{3}{*}{20\%}
      & ML-CLOCK (Usulan) & -0.20\% & 10\% \\
      & Optimal Zipf & \textbf{-1.67\%} & 15\% \\
      & Offline CLOCK & -3.50\% & 43\% \\
      \midrule
      \multirow{3}{*}{40\%}
      & ML-CLOCK (Usulan) & +0.15\% (Regresi) & 8\% \\
      & Optimal Zipf & \textbf{-3.30\%} & 17\% \\
      & Offline CLOCK & -8.00\% & 43\% \\
      \bottomrule
    \end{tabular}%
  }
\end{table}

\section{Analisis Karakteristik Algoritma Pembanding}

Untuk menempatkan kinerja ML-CLOCK dalam perspektif yang objektif,
karakteristik fundamental dari ketiga algoritma pembanding harus
dipahami sebagai berikut:

\subsection{Optimal Zipf Promotion (Oracle Heuristik)}
Strategi ini bertindak sebagai algoritma \textit{Oracle} yang "curang" karena
memanfaatkan \textbf{pengetahuan masa depan yang lengkap} mengenai
distribusi data.
Algoritma ini secara eksplisit membaca identitas objek
(\texttt{objectId}) dan hanya mempromosikan
objek yang diketahui berada di puncak kurva popularitas Zipfian.
Dalam skenario dunia nyata, sistem tidak memiliki akses ke "kunci jawaban"
berupa distribusi probabilitas global ini.
Oleh karena itu, Optimal Zipf memberikan batas kinerja heuristik yang
sangat tinggi
namun tidak praktis untuk implementasi nyata.

\subsection{Offline CLOCK (Batas Atas Teoretis)}
Offline CLOCK merupakan representasi dari kesempurnaan teoretis
(mirip dengan algoritma Belady's OPT).
Algoritma ini bekerja dengan cara memindai seluruh \textit{trace}
akses di masa depan sebelum simulasi dimulai.
Keputusan promosi diambil dengan logika deterministik sempurna:
sebuah objek hanya akan dipromosikan jika dan hanya jika objek tersebut
dipastikan akan diakses kembali (\textit{hit}) sebelum waktu
penggusurannya tiba.
Kesenjangan kinerja yang masif antara ML-CLOCK dan Offline CLOCK
(43\% reduksi promosi)
menunjukkan batas fisik maksimal yang bisa dicapai jika prediksi masa
depan memiliki akurasi 100\%.

\subsection{CLOCK Standar (Baseline)}
Algoritma ini mewakili pendekatan konvensional tanpa kecerdasan buatan.
Ia bekerja secara reaktif murni berdasarkan prinsip \textit{Recency} (kebaruan).
Kelemahan fatalnya adalah ketidakmampuan membedakan objek populer dengan objek
yang hanya lewat sekali (\textit{one-hit wonders}).
Akibatnya, CLOCK Standar sering melakukan promosi yang sia-sia
(\textit{wasted promotions}),
yang memboroskan siklus penulisan memori dan mencemari \textit{cache}
dengan data sampah.

\section{Evaluasi Model Machine Learning}

Meskipun kalah dari strategi Oracle dan Offline yang memiliki
pengetahuan masa depan,
model ML-CLOCK menunjukkan karakteristik kinerja yang spesifik
sebagai solusi \textit{online}:

\begin{enumerate}
  \item \textbf{Efektivitas Terbatas pada Cache Kecil:}
    Pada ukuran 1\% hingga 20\%, model berhasil memberikan kontribusi positif
    meskipun marjinal. Penurunan \textit{miss ratio} sebesar 0.05\% -- 0.20\%
    membuktikan bahwa model mampu belajar membedakan objek \textit{wasted}.

  \item \textbf{Kegagalan Generalisasi pada Cache Besar:}
    Pada ukuran 40\%, terjadi regresi kinerja yang signifikan.
    Ambang batas (\textit{threshold}) klasifikasi model mungkin kurang adaptif
    untuk kapasitas penyimpanan yang sangat besar.

  \item \textbf{Peran Vital Bobot dan Threshold:}
    Tanpa penerapan bobot \textit{loss} kecil (0.5) untuk kelas \texttt{wasted}
    dan ambang batas prediksi tinggi (0.6 -- 0.8), model menjadi terlalu agresif
    menolak promosi (\textit{aggressive eviction}).
    Mekanisme kontrol ini mencegah model membuang terlalu banyak objek potensial
    yang akan menghancurkan kinerja \textit{miss ratio}.
\end{enumerate}

% --- Template Gambar Cache 1% ---
\begin{figure}[htbp]
  \centering
  % Ganti 'nama_file_gambar_1.png' dengan file Anda
  \includegraphics[width=0.8\textwidth]{gambar/hasil/cache_size_0_01.png}
  \caption{Perbandingan Miss Ratio dan Total Promosi pada Ukuran Cache 1\%}
  \label{fig:res_1pct}
\end{figure}

% --- Template Gambar Cache 10% ---
\begin{figure}[htbp]
  \centering
  % Ganti 'nama_file_gambar_10.png' dengan file Anda
  \includegraphics[width=0.8\textwidth]{gambar/hasil/cache_size_0_1.png}
  \caption{Perbandingan Miss Ratio dan Total Promosi pada Ukuran Cache 10\%}
  \label{fig:res_10pct}
\end{figure}

% --- Template Gambar Cache 20% ---
\begin{figure}[htbp]
  \centering
  % Ganti 'nama_file_gambar_20.png' dengan file Anda
  \includegraphics[width=0.8\textwidth]{gambar/hasil/cache_size_0_2.png}
  \caption{Perbandingan Miss Ratio dan Total Promosi pada Ukuran Cache 20\%}
  \label{fig:res_20pct}
\end{figure}

% --- Template Gambar Cache 40% ---
\begin{figure}[htbp]
  \centering
  % Ganti 'nama_file_gambar_40.png' dengan file Anda
  \includegraphics[width=0.8\textwidth]{gambar/hasil/cache_size_0_4.png}
  \caption{Perbandingan Miss Ratio dan Total Promosi pada Ukuran Cache 40\%}
  \label{fig:res_40pct}
\end{figure}

\section{Analisis Fitur dan Konfigurasi Model}

Model terbaik yang dihasilkan menggunakan kombinasi fitur frekuensi
(\texttt{lifetime\_freq}, \texttt{cache\_freq}) dan kebaruan
(\texttt{vtime\_since\_access}).
Penting dicatat bahwa konfigurasi ambang batas tinggi (0.6 -- 0.8) berfungsi
sebagai mekanisme pertahanan (\textit{safeguard}).
Model memiliki bias inheren untuk mengklasifikasikan objek sebagai
\texttt{wasted}.
Tanpa ambang batas yang ketat ini, sistem akan mengalami \textit{over-eviction},
di mana penghematan promosi mungkin melonjak drastis, namun dibayar mahal
dengan degradasi performa layanan (\textit{miss ratio}).
