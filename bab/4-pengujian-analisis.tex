\chapter{PENGUJIAN DAN ANALISIS}
\label{chap:pengujian_analisis}

Bab ini menyajikan hasil yang dicapai selama penelitian, dimulai dari
hasil yang diharapkan, temuan-temuan dari hasil pengujian awal, serta
analisis terhadap data tersebut untuk memvalidasi pendekatan yang diusulkan.

\section{Hasil yang Diharapkan}
\label{sec:hasil_yang_diharapkan}

Ekspektasi utama dari penelitian ini adalah bahwa model
\textit{machine learning} yang dikembangkan akan mampu secara
signifikan mengurangi jumlah total promosi objek di dalam algoritma
CLOCK. Pengurangan ini diharapkan dapat dicapai tanpa mengorbankan
kinerja utama \textit{cache}, yaitu dengan tidak meningkatkan
\textit{miss ratio} secara berarti. Dengan kata lain, tujuannya
adalah mencapai efisiensi manajemen \textit{cache} yang lebih tinggi
dengan membuang promosi yang tidak perlu, sambil tetap mempertahankan
data yang relevan di dalam \textit{cache}.

\section{Hasil Pengujian Awal}
\label{sec:hasil_pengujian_awal}

Hingga tahap ini, penulis telah berhasil menyelesaikan beberapa
tonggak penting. Perangkat lunak simulasi untuk akuisisi data dari
\textit{trace} telah berhasil dikembangkan. Selain itu, skrip Python
untuk memproses dan memvisualisasikan data yang dihasilkan juga telah
dibuat, memungkinkan analisis yang lebih cepat dan interaktif.

\subsection{Hasil Eksperimen Offline CLOCK}
Eksperimen awal dilakukan dengan menggunakan algoritma Offline CLOCK
pada \textit{trace} Meta Key-Value untuk menetapkan batas atas
kinerja. Hasil dari simulasi ini, seperti yang ditunjukkan pada
Gambar \ref{fig:hasil_offline_clock}, menunjukkan adanya penurunan
drastis pada jumlah promosi dan juga pada \textit{miss ratio}.
Penurunan yang signifikan ini merupakan indikasi kuat bahwa pada
implementasi standar algoritma CLOCK, terjadi banyak sekali promosi
yang sia-sia (\textit{wasted promotions}). Promosi ini tidak hanya
membuang sumber daya komputasi, tetapi juga berpotensi "mengotori"
\textit{cache} dengan objek-objek yang sebenarnya tidak akan diakses kembali.

\begin{figure}[H]
  \centering
  \includegraphics[scale=0.4]{gambar/offline.png}
  \caption{Grafik hasil dari simulasi Offline CLOCK pada
  \textit{trace} Meta Key-Value.}
  \label{fig:hasil_offline_clock}
\end{figure}

\subsection{Hasil Pengembangan Model Awal}
Pada tahap pengembangan model, penulis telah melakukan eksperimen
awal dengan model regresi logistik (\textit{logistic regression}).
Model ini dipilih karena kesederhanaannya dan kemampuannya untuk
diinterpretasikan. Bobot dan ambang batas (\textit{threshold}) dari
model ini telah disesuaikan untuk mendapatkan hasil yang optimal pada
data validasi. Hasil dari model ini dapat dilihat pada Gambar
\ref{fig:hasil_model_lr}.

\begin{figure}[H]
  \centering
  \includegraphics[scale=0.4]{gambar/res.png}
  \caption{Grafik perbandingan hasil dari model regresi logistik.}
  \label{fig:hasil_model_lr}
\end{figure}

Gambar \ref{fig:hasil_model_lr} menunjukkan perbandingan rata-rata
pengurangan jumlah promosi dan perubahan pada \textit{miss ratio}
setelah penerapan model. Seperti yang terlihat, jumlah promosi
berhasil dikurangi secara signifikan, sekitar \textbf{10\%},
sementara pada saat yang sama, \textit{miss ratio} tidak terlalu
terpengaruh dan tetap stabil. Hasil awal ini sangat menjanjikan dan
selaras dengan tujuan utama penelitian, yaitu mengurangi promosi
tanpa mengorbankan \textit{hit ratio}.

