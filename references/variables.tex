% Atur variabel berikut sesuai namanya

% nama
\newcommand{\name}                  {Muhammad Haekal Muhyidin Al-Araby}
\newcommand{\authorname}            {Al-Araby, Muhammad Haekal Muhyidin}
\newcommand{\nickname}              {Haekal}
\newcommand{\advisor}               {Reza Fuad Rachmadi, S.T., M.T., Ph.D}
\newcommand{\coadvisor}             {Dr. Arief Kurniawan, S.T., M.T}
\newcommand{\examinerone}           {Dr. Eko Mulyanto Yuniarno, S.T., M.T}
\newcommand{\examinertwo}           {Dr. Diah Puspito Wulandari, S.T., M.Sc}
\newcommand{\examinerthree}         {Yusril Izza, S.T., M.Comp}
\newcommand{\headofdepartment}      {Dr. Arief Kurniawan, S.T., M.T}

% identitas
\newcommand{\nrp}                   {5024 22 1030}
\newcommand{\advisornip}            {19850403 201212 1 001}
\newcommand{\coadvisornip}          {19740907 200212 1 001}
\newcommand{\examineronenip}        {19680601 199512 1 009}
\newcommand{\examinertwonip}        {19801219 200501 2 001}
\newcommand{\examinerthreenip}      {1992202511059}
\newcommand{\headofdepartmentnip}   {19740907 200212 1 001}

% judul
\newcommand{\tatitle}{PENERAPAN MODEL MACHINE LEARNING UNTUK
MENINGKATKAN EFISIENSI ALGORITMA CLOCK PADA CACHE REPLACEMENT}
\newcommand{\engtatitle}{\emph{APPLICATION OF MACHINE LEARNING MODELS
TO IMPROVE THE EFFICIENCY OF THE CLOCK ALGORITHM IN CACHE REPLACEMENT}}

% tempat
\newcommand{\place}{Surabaya}

% jurusan
\newcommand{\studyprogram}{Teknik Komputer}
\newcommand{\engstudyprogram}{Computer Engineering}

% fakultas
\newcommand{\faculty}{Teknologi Elektro dan Informatika Cerdas}
\newcommand{\engfaculty}{Electrical Engineering and Intelligent Informatics}

% singkatan fakultas
\newcommand{\facultyshort}{FTEIC}
\newcommand{\engfacultyshort}{ELECTICS}

% departemen
\newcommand{\department}{Teknik Komputer}
\newcommand{\engdepartment}{Computer Engineering}

% kode mata kuliah
\newcommand{\coursecode}{EC234701}
