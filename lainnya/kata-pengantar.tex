\begin{center}
  \large\textbf{KATA PENGANTAR}
\end{center}

\addcontentsline{toc}{chapter}{KATA PENGANTAR}

\vspace{2ex}

Puji syukur penulis panjatkan kehadirat Tuhan Yang Maha Esa, yang
atas rahmat dan karunia-Nya telah memberikan kekuatan dan kelancaran
sehingga penulis dapat menyelesaikan Laporan Tugas Akhir yang
berjudul "\textbf{\indotatitle{}}" dengan baik dan tepat waktu.

Penelitian dan penyusunan Laporan Tugas Akhir ini merupakan salah
satu syarat untuk memperoleh gelar Sarjana Teknik pada Program Studi
\studyprogram{}, Departemen \department{}, Fakultas \faculty{},
Institut Teknologi Sepuluh Nopember. Proses ini tidak akan berjalan
lancar tanpa adanya bimbingan, dukungan, dan doa dari berbagai pihak.
Oleh karena itu, pada kesempatan ini, penulis ingin mengucapkan
terima kasih yang tulus kepada:

\begin{enumerate}[nolistsep]
  \item Keluarga tercinta, terutama Ayah, Ibu, dan seluruh saudara,
    yang telah memberikan dukungan moral, materiel, serta doa yang
    tiada henti sepanjang perjalanan studi penulis.
  \item Bapak \advisor{}, selaku Dosen Pembimbing I, dan Bapak
    \coadvisor{}, selaku Dosen Pembimbing II, yang telah dengan sabar
    memberikan bimbingan, arahan, dan masukan yang sangat berharga
    sejak awal hingga akhir penyusunan tugas akhir ini.
  \item Bapak \headofdepartment{}, selaku Kepala Departemen
    \department{}, serta seluruh jajaran dosen dan staf di lingkungan
    departemen yang telah memberikan bekal ilmu dan kemudahan
    administrasi selama masa perkuliahan.
  \item Seluruh sahabat dan rekan-rekan seperjuangan di Departemen
    \department{} yang telah menjadi teman diskusi, memberikan
    semangat, dan saling membantu dalam suka maupun duka selama
    menempuh pendidikan.
  \item Semua pihak yang tidak dapat penulis sebutkan satu per satu,
    yang telah memberikan kontribusi dan dukungan dalam penyelesaian
    laporan ini.
\end{enumerate}

Penulis menyadari bahwa Laporan Tugas Akhir ini masih jauh dari
kesempurnaan. Oleh karena itu, segala bentuk kritik dan saran yang
membangun akan penulis terima dengan lapang dada demi perbaikan di
masa mendatang. Akhir kata, semoga laporan ini dapat memberikan
manfaat dan kontribusi bagi perkembangan ilmu pengetahuan, khususnya
di bidang \textit{caching system} dan \textit{machine learning}.

\begin{flushright}
  \begin{tabular}[b]{c}
    \place{}, \MONTH{} \the\year{} \\
    \vspace{5ex}
    \\
    \name{}
  \end{tabular}
\end{flushright}

