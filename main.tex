% Judul dokumen
\title{Buku Tugas Akhir ITS}
\author{, Muhammad Haekal Muhyidin Al-Araby}

% Pengaturan ukuran teks dan bentuk halaman dua sisi
\documentclass[12pt,twoside]{report}

% Pengaturan ukuran halaman dan margin
\usepackage[a4paper,top=30mm,left=30mm,right=20mm,bottom=25mm]{geometry}

% Pengaturan ukuran spasi
\usepackage[singlespacing]{setspace}

% Pengaturan detail pada file PDF
\usepackage[pdfauthor={\@author},bookmarksnumbered,pdfborder={0 0 0}]{hyperref}
\usepackage{xcolor}
% Pengaturan jenis karakter
\usepackage[utf8]{inputenc}

% Pengaturan pewarnaan
\usepackage[table,xcdraw]{xcolor}

% Pengaturan kutipan artikel
\usepackage[style=ieee, backend=biber]{biblatex}

% Package lainnya
\usepackage{changepage}
\usepackage{enumitem}
\usepackage{eso-pic}
\usepackage{txfonts} % Font times
\usepackage{etoolbox}
\usepackage{graphicx}
\usepackage{lipsum}
\usepackage{longtable}
\usepackage{tabularx}
\usepackage{wrapfig}
\usepackage{float}

% Definisi untuk "Hati ini sengaja dikosongkan"
\patchcmd{\cleardoublepage}{\hbox{}}{
  \thispagestyle{empty}
  \vspace*{\fill}
  \begin{center}\textit{[Halaman ini sengaja dikosongkan]}
  \end{center}
\vfill}{}{}

% Pengaturan penomoran halaman
\usepackage{fancyhdr}
\fancyhf{}
\renewcommand{\headrulewidth}{0pt}
\pagestyle{fancy}
\fancyfoot[LE,RO]{\thepage}
\patchcmd{\chapter}{plain}{fancy}{}{}
\patchcmd{\chapter}{empty}{plain}{}{}

% Command untuk bulan
\newcommand{\MONTH}{%
  \ifcase\the\month
  \or Januari% 1
  \or Februari% 2
  \or Maret% 3
  \or April% 4
  \or Mei% 5
  \or Juni% 6
  \or Juli% 7
  \or Agustus% 8
  \or September% 9
  \or Oktober% 10
  \or November% 11
  \or Desember% 12
\fi}
\newcommand{\ENGMONTH}{%
  \ifcase\the\month
  \or January% 1
  \or February% 2
  \or March% 3
  \or April% 4
  \or May% 5
  \or June% 6
  \or July% 7
  \or August% 8
  \or September% 9
  \or October% 10
  \or November% 11
  \or December% 12
\fi}

% Pengaturan format judul bab
\usepackage{titlesec}
\titleformat{\chapter}[display]{\bfseries\Large}{CHAPTER
\centering\Roman{chapter}}{0ex}{\vspace{0ex}\centering}
\titleformat{\section}{\bfseries\large}{\MakeUppercase{\thesection}}{1ex}{\vspace{1ex}}
\titleformat{\subsection}{\bfseries\large}{\MakeUppercase{\thesubsection}}{1ex}{}
\titleformat{\subsubsection}{\bfseries\large}{\MakeUppercase{\thesubsubsection}}{1ex}{}
\titlespacing{\chapter}{0ex}{0ex}{4ex}
\titlespacing{\section}{0ex}{1ex}{0ex}
\titlespacing{\subsection}{0ex}{0.5ex}{0ex}
\titlespacing{\subsubsection}{0ex}{0.5ex}{0ex}

% Atur variabel berikut sesuai namanya

% nama
\newcommand{\name}                  {Muhammad Haekal Muhyidin Al-Araby}
\newcommand{\authorname}            {Al-Araby, Muhammad Haekal Muhyidin}
\newcommand{\nickname}              {Haekal}
\newcommand{\advisor}               {Reza Fuad Rachmadi, S.T., M.T., Ph.D}
\newcommand{\coadvisor}             {Dr. Arief Kurniawan, S.T., M.T}
\newcommand{\examinerone}           {Dr. Eko Mulyanto Yuniarno, S.T., M.T}
\newcommand{\examinertwo}           {Dr. Diah Puspito Wulandari, S.T., M.Sc}
\newcommand{\examinerthree}         {Arta Kusuma Hernanda, S.T., M.T}
\newcommand{\examinerfour}          {Yusril Izza, S.T., M.Comp}
\newcommand{\headofdepartment}      {Dr. Arief Kurniawan, S.T., M.T}

% identitas
\newcommand{\nrp}                   {5024 22 1030}
\newcommand{\advisornip}            {19850403 201212 1 001}
\newcommand{\coadvisornip}          {19740907 200212 1 001}
\newcommand{\examineronenip}        {19680601 199512 1 009}
\newcommand{\examinertwonip}        {19801219 200501 2 001}
\newcommand{\examinerthreenip}      {1996202311024}
\newcommand{\examinerfournip}       {1992202511059}
\newcommand{\headofdepartmentnip}   {19740907 200212 1 001}

% judul
\newcommand{\tatitle}{PENERAPAN MODEL MACHINE LEARNING UNTUK
MENINGKATKAN EFISIENSI ALGORITMA CLOCK PADA CACHE REPLACEMENT}
\newcommand{\engtatitle}{\emph{APPLICATION OF MACHINE LEARNING MODELS
TO IMPROVE THE EFFICIENCY OF THE CLOCK ALGORITHM IN CACHE REPLACEMENT}}

% tempat
\newcommand{\place}{Surabaya}

% jurusan
\newcommand{\studyprogram}{Teknik Komputer}
\newcommand{\engstudyprogram}{Computer Engineering}

% fakultas
\newcommand{\faculty}{Teknologi Elektro dan Informatika Cerdas}
\newcommand{\engfaculty}{Electrical Engineering and Intelligent Informatics}

% singkatan fakultas
\newcommand{\facultyshort}{FTEIC}
\newcommand{\engfacultyshort}{ELECTICS}

% departemen
\newcommand{\department}{Teknik Komputer}
\newcommand{\engdepartment}{Computer Engineering}

% kode mata kuliah
\newcommand{\coursecode}{EC234701}


% Tambahkan format tanda hubung yang benar di sini
\hyphenation{
  ro-ket
  me-ngem-bang-kan
  per-hi-tu-ngan
  tek-no-lo-gi
  me-la-ku-kan
  ber-so-si-al-i-sa-si
}

% Menambahkan resource daftar pustaka
\addbibresource{references/references.bib}

% Pengaturan format potongan kode
\usepackage{listings}
\definecolor{comment}{RGB}{0,128,0}
\definecolor{string}{RGB}{255,0,0}
\definecolor{keyword}{RGB}{0,0,255}
\lstdefinestyle{codestyle}{
  commentstyle=\color{comment},
  stringstyle=\color{string},
  keywordstyle=\color{keyword},
  basicstyle=\footnotesize\ttfamily,
  numbers=left,
  numberstyle=\tiny,
  numbersep=5pt,
  frame=lines,
  breaklines=true,
  prebreak=\raisebox{0ex}[0ex][0ex]{\ensuremath{\hookleftarrow}},
  showstringspaces=false,
  upquote=true,
  tabsize=2,
}
\lstset{style=codestyle}

\newcommand{\todo}[1]{\textcolor{red}{\textbf{todo: #1}}}

% Isi keseluruhan dokumen
\begin{document}

% Sampul luar Bahasa Indonesia
\newcommand\covercontents{sampul/konten-id.tex}
\input{sampul/sampul-luar.tex}

% Atur ulang penomoran halaman
\setcounter{page}{1}

% Sampul dalam Bahasa Indonesia
\renewcommand\covercontents{sampul/konten-id.tex}
\input{sampul/sampul-luar-tipis.tex}
\clearpage
\cleardoublepage

% Sampul dalam Bahasa Inggris
\renewcommand\covercontents{sampul/konten-en.tex}
\input{sampul/sampul-luar-tipis.tex}
\cleardoublepage

% Label tabel dan gambar dalam bahasa indonesia
\renewcommand{\figurename}{Gambar}
\renewcommand{\tablename}{Tabel}

% Pengaturan ukuran indentasi paragraf
\setlength{\parindent}{2em}

% Pengaturan ukuran spasi paragraf
\setlength{\parskip}{1ex}

% Lembar pengesahan
\input{lainnya/lembar-pengesahan.tex}
\cleardoublepage
\input{lainnya/lembar-pengesahan-en.tex}
\cleardoublepage

% Pernyataan keaslian
\input{lainnya/pernyataan-keaslian.tex}
\cleardoublepage
\input{lainnya/pernyataan-keaslian-en.tex}
\cleardoublepage

% Nomor halaman pembuka dimulai dari sini
\pagenumbering{roman}

% Abstrak Bahasa Indonesia
\input{abstrak/abstrak-id.tex}
\cleardoublepage

% Abstrak Bahasa Inggris
\input{abstrak/abstrak-en.tex}
\cleardoublepage

% Kata pengantar
\input{lainnya/kata-pengantar.tex}
\cleardoublepage

% Daftar isi
\renewcommand*\contentsname{DAFTAR ISI}
\addcontentsline{toc}{chapter}{\contentsname}
\tableofcontents
\cleardoublepage

% Daftar gambar
\renewcommand*\listfigurename{DAFTAR GAMBAR}
\addcontentsline{toc}{chapter}{\listfigurename}
\listoffigures
\cleardoublepage

% Daftar tabel
\renewcommand*\listtablename{DAFTAR TABEL}
\addcontentsline{toc}{chapter}{\listtablename}
\listoftables
\cleardoublepage

% Nomor halaman isi dimulai dari sini
\pagenumbering{arabic}

% Bab 1 pendahuluan
\chapter{INTRODUCTION}
\label{chap:intro}
This research is driven by the advancement of modern computing
technology, which is in dire need of more advanced techniques for
object caching\todo{cite}. Machine learning\todo{cite} has recently
become a hot topic in computer science research\todo{cite}. Numerous
studies have been carried out to understand the extent of machine
learning capabilities in the field of caching\todo{cite}. This
research investigates how to implement machine learning in cache
systems and how it compares to popular algorithms.

\section{Background}
\label{sec:background}

The advancement of computing technology has progressed at an
incredible speed over the past decade\todo{cite}. In the early
days, computers were limited to single core processor\todo{cite}.
This limitation forced engineers and
programmers to design software that was optimized for
single-threaded execution. As performance improvements increases began
to plateau due to physical and thermal
limitations\todo{cite}, the industry shifted toward multi-core and
parallel architectures. This transition introduced new challenges in
software design, requiring developers to adopt concurrent and
parallel programming models to fully exploit the capabilities of
modern hardware. Consequently, understanding and optimizing
multi-threaded applications has become an essential aspect of modern
computing research and development.

Caching has long been an integral part of computer
systems\todo{cite}. It plays a crucial role in improving system
performance. Over the years, various caching strategies have been
developed to optimize data retrieval efficiency across different
environments, from operating systems and web servers to
distributed systems and cloud infrastructures\todo{cite}.

Multi-core architectures, which have become increasingly popular over
the years, also present significant challenges to existing caching
techniques. In such environments, multiple cores often access and
modify shared cache data concurrently, leading to potential race
conditions and data inconsistency\todo{cite}. To maintain
correctness, most traditional cache implementations emphasize the use
of locks during cache insertion, promotion, and modification
operations\todo{cite}. However, excessive locking can introduce
contention and reduce parallelism, ultimately limiting system
performance on modern multi-core processors\todo{cite}.

Various cache replacement policies have been proposed to address
different design goals. The Least Recently Used (LRU) policy is
commonly adopted due to its ability to achieve a low miss ratio by
retaining frequently accessed data in the cache\todo{cite}. However,
maintaining precise recency information requires frequent updates and
synchronization, which increases locking overhead. On the other hand,
simpler policies such as First-In First-Out (FIFO) are often
preferred for their scalability and lower synchronization cost, as
they require no locking\todo{cite}.

To overcome the limitations of both LRU and FIFO, the CLOCK algorithm
was developed as an approximation to LRU that maintains high
scalability\todo{cite}. CLOCK achieves this by using a circular
buffer of cache entries, each associated with a reference bit that
indicates recent access. Instead of maintaining a fully ordered list
like LRU, the algorithm advances a “clock hand” to find entries for
replacement, clearing reference bits along the way. This design
significantly reduces the need for excessive locking and frequent
updates, while still preserving the benefits of
LRU. As a result, CLOCK offers a practical balance between achieving
a low miss ratio and maintaining the scalability and simplicity
characteristic of FIFO\todo{cite}.

However, despite its advantages, the CLOCK algorithm still suffers
from unnecessary promotions~\cite{chen2025lazy}. Because each access
sets the reference bit regardless of access frequency or importance,
even one-time or infrequent accesses can prevent an entry from being
evicted. As a result, the cache may retain items that provide little
benefit while displacing those with higher reuse potential. This leads to
inefficiencies in cache utilization.

Recent developments in machine learning have opened possibilities to
use it as an additional decision-making component to reduce
unnecessary promotions in cache systems. By learning from
access patterns and identifying objects that are unlikely to be
reused, machine learning models can help prevent the cache from
promoting entries that provide little benefit. This approach allows
the caching mechanism to make more selective and informed decisions
about which objects deserve to remain in the cache. As a result,
machine learning can complement traditional algorithms like CLOCK by
improving efficiency, reducing management overhead, and adapting
dynamically to changing workload characteristics.

\section{Problem Statement}
\label{sec:problem}

Based on the discussion in Section \ref{sec:background}, the main
problem addressed in this research is how to integrate machine
learning techniques into the CLOCK cache replacement algorithm in
order to reduce unnecessary promotions and improve overall system
efficiency. Specifically, this study aims to explore how data-driven
decision-making can help the CLOCK algorithm better approximate
optimal replacement behavior. By doing so, the research seeks to
enhance both the accuracy and scalability of CLOCK-based cache management.

\section{Objectives}
\label{sec:Tujuan}

The main objectives of this research are as follows:

\begin{enumerate}[nolistsep]
  \item To integrate machine learning models into the CLOCK cache
    replacement algorithm to enhance its decision-making process.
  \item To evaluate the efficiency and effectiveness of the machine
    learning–based CLOCK algorithm in reducing promotions and
    improving overall cache performance.
\end{enumerate}

\section{Scope of the Study}
\label{sec:scope}

This study focuses on the integration of machine learning techniques
into the CLOCK cache replacement algorithm to improve its efficiency
and reduce unnecessary promotions. The research is limited to
software-based cache simulation environments rather than hardware
implementations. The study evaluates performance metrics such as hit
ratio and promotion rate.

The scope does not include the development of new machine learning
architectures or the comparison with advanced reinforcement
learning–based caching systems. Instead, the research emphasizes
analyzing how existing lightweight models can enhance decision-making
in CLOCK without significantly increasing complexity. The results and
conclusions are applicable primarily to cache management systems
operating in single-node or multi-core environments.

\section{Organization of the Report}
\label{sec:sistematikapenulisan}

This final project report is divided into five chapters as follows:

\begin{enumerate}[nolistsep]

  \item \textbf{CHAPTER I Introduction}

    This chapter presents the background of the study, the main
    problems to be addressed, the objectives of the research, and the
    scope of the study. It provides an overview of the motivation and
    significance of the research topic.

    \vspace{2ex}

  \item \textbf{CHAPTER II Literature Review}

    This chapter discusses previous works and related studies that
    serve as the foundation of this research. It highlights existing
    cache replacement algorithms, the development of the CLOCK
    algorithm, and prior applications of machine learning in caching systems.

    \vspace{2ex}

  \item \textbf{CHAPTER III System Design and Implementation}

    This chapter describes the design and implementation of the
    proposed system. It explains the overall system architecture,
    data flow, machine learning model integration, and the process of
    adapting the model to the CLOCK cache replacement algorithm.

    \vspace{2ex}

  \item \textbf{CHAPTER IV Testing and Analysis}

    This chapter presents the testing procedures, experimental setup,
    and evaluation metrics used to measure system performance. It
    also provides an analysis of the results, comparing the proposed
    approach with baseline methods in terms of efficiency and
    promotion reduction.

    \vspace{2ex}

  \item \textbf{CHAPTER V Conclusion}

    This chapter summarizes the findings and contributions of the
    research. It also discusses possible improvements and
    recommendations for future work related to machine learning–based
    cache management.

\end{enumerate}

\cleardoublepage

% Bab 2 tinjauan pustaka
\input{bab/2-tinjauan-pustaka.tex}
\cleardoublepage

% Bab 3 desain dan implementasi
\input{bab/3-desain-implementasi.tex}
\cleardoublepage

% Bab 4 pengujian dan analisis
\input{bab/4-pengujian-analisis.tex}
\cleardoublepage

% Bab 5 penutup
\input{bab/5-penutup.tex}
\cleardoublepage

\chapter*{DAFTAR PUSTAKA}
\addcontentsline{toc}{chapter}{DAFTAR PUSTAKA}
\renewcommand\refname{}
\vspace{2ex}
\renewcommand{\bibname}{}
\begingroup
\def\chapter*#1{}
\printbibliography
\endgroup
\cleardoublepage

% Biografi penulis
\input{lainnya/biografi-penulis.tex}
\cleardoublepage

\end{document}
